\documentclass[a4paper]{jarticle}
\setlength{\textwidth}{170mm}       % テキストの幅
\setlength{\textheight}{260mm}      % テキストの高さ
\setlength{\oddsidemargin}{-5mm}    % 偶数ページの左マージン
\setlength{\evensidemargin}{0mm}    % 奇数ページの左マージン
\setlength{\topmargin}{-25mm}       % 上のマージン
\usepackage{listings}
\usepackage[dvipdfmx]{graphicx}
\usepackage{bm}
\usepackage{amsmath,amssymb}
\usepackage{mathtools}
\mathtoolsset{showonlyrefs=true}
\usepackage{ascmac}
\usepackage{fancybox}
\usepackage[dvipdfmx, bookmarkstype=toc, colorlinks=false, pdfborder={0 0 0}, bookmarks=true, bookmarksnumbered=true]{hyperref}
\usepackage{pxjahyper}
\newcommand{\argmax}{\mathop{\rm arg~max}\limits}
\newcommand{\argmin}{\mathop{\rm arg~min}\limits}
\lstset{
  basicstyle={\ttfamily},
  identifierstyle={\small},
  commentstyle={\smallitshape},
  keywordstyle={\small\bfseries},
  ndkeywordstyle={\small},
  stringstyle={\small\ttfamily},
  frame={tb},
  breaklines=true,
  columns=[l]{fullflexible},
  numbers=left,
  xrightmargin=0zw,
  xleftmargin=3zw,
  numberstyle={\scriptsize},
  stepnumber=1,
  numbersep=1zw,
  lineskip=-0.5ex
}


\begin{document}
\tableofcontents
\newpage
\hypertarget{algorithm}{\section{アルゴリズム}}
\hypertarget{allserch}{\subsection{全探索}}
\hypertarget{dfs}{\subsubsection{深さ優先探索}}
ある状態からはじめ、遷移できなくなる状態まで遷移し、遷移できなくなったら1つ前の状態に戻ることを繰り返す。\\
性質上、再帰関数で書くことが可能。DFSよりも実装が楽。
\hypertarget{bfs}{\subsubsection{幅優先探索}}
初めの状態から近い状態に遷移していく。\\
最短路を求める場合はこちらを採用。
\hypertarget{dp}{\subsection{動的計画法}}
\hypertarget{napsack}{\subsubsection{ナップサック}}
\[
dp[i][j]=上からi番目の荷物を使って、重さjで価値が最大
\]
となるように$dp$を更新していく
\hypertarget{digitdp}{\subsubsection{桁DP}}
\[
dp[i][j][k]=上からi桁目までで、条件jを満し、Nと同じか未満か(k=0:同じ,k=1:未満)
\]
となるように$dp$を更新していく
\hypertarget{glaph}{\subsection{グラフ}}

\hypertarget{math}{\section{数学的知識}}
\hypertarget{felmat}{\subsection{フェルマーの小定理}}
\hypertarget{combination}{\subsection{組み合わせ}}
\begin{itembox}[l]{組み合わせの公式}
\[
{}_nC_k=\frac{n!}{k!\cdot (n-k)!}
\]
\end{itembox}

\begin{itembox}[l]{組み合わせの和}
\begin{align}
\sum_{k=0}^n{}_nC_k&=2^n & \sum_{k=0}^nk\cdot{}_nC_k&=n2^{n-1}
\end{align}
\end{itembox}
\newpage
\hypertarget{abc}{\section{AtCoder Beginner Contest}}
\hypertarget{150}{\subsection{150}}
\hypertarget{151}{\subsection{151}}
\hypertarget{152}{\subsection{152}}
\hypertarget{153}{\subsection{153}}
\hypertarget{153e}{\subsubsection{E}}
\noindent
\textbf{問題文}\\
$トキはモンスターと戦っています。モンスターの体力は Hです。\\
トキは N種類の魔法が使え、i番目の魔法を使うと、モンスターの体力を A_i減らすことができますが、トキの魔力を B_i消耗します。\\
同じ魔法は何度でも使うことができます。魔法以外の方法でモンスターの体力を減らすことはできません。
モンスターの体力を 0以下にすればトキの勝ちです。\\
トキがモンスターに勝つまでに消耗する魔力の合計の最小値を求めてください。$\\\\
\textbf{制約}\\
$1\le H\le10^4\\
1\le N\le10^3\\
1\le A_i\le10^4\\
1\le B_i\le10^4$\\
入力中のすべての値は整数である。
\\\\
\textbf{方針}\\
DPを使用する。

\[
dp[i][j]=i番目までの魔法で、体力jを減らすのに必要な最小魔力
\]
となるように$dp$を更新していく。



\hypertarget{154}{\subsection{154}}
\hypertarget{154e}{\subsubsection{E}}
\noindent
\textbf{問題文}\\
$1以上 N以下の整数であって、 10進法で表したときに、0でない数字がちょうど K個あるようなものの個数を求めてください。$
\\\\
\textbf{制約}\\
$1\le N<10^{100}\\
1\le K\le3$
\\\\
\textbf{方針}\\
桁DPを使用する。
\[
dp[i][j][k]=上からi桁目までで、0でない数字がj個あり、Nと同じか未満か(k=0:同じ,k=1:未満)
\]
となるように$dp$を更新していく

\newpage
\hypertarget{edcp}{\section{Educational DP Contest}}

\newpage
\hypertarget{tyuiten}{\section{注意点}}
\hypertarget{order}{\subsection{計算量}}
\noindent
1秒あたりにできる計算量は約$10^7$くらい。\\
これを踏まえるとNあたりにできる最大計算量は以下の表の通り。
\begin{table}[htbp]
\centering
  \begin{tabular}{|c|c|}\hline
  N&計算量\\\hline\hline
  $10^6$	&$O(N\log N),O(N)$\\\hline
  $10^5$&$O(N \log^2 N),O(N\log  N)$\\\hline
  $3000$&$O(N^2)$\\\hline
  $300$&$O(N^3)(単純な処理)$\\\hline
  $100$&$O(N^3)$\\\hline
  $50$&$O(N^4)$\\\hline
  $10$&$O(N*2^N),O(2^N)$\\\hline
  	
  \end{tabular}
\end{table}
\hypertarget{mod}{\subsection{mod}}
\noindent
\textbf{足し算・掛け算}\\
演算をするごとに$\bmod1\rm e9+7をする。$\\\\
\textbf{引き算}\\
$最後に1\rm e9+7を足した上で\bmod1\rm e9+7する。$

\end{document}